\usepackage{environ}   % A better command for defining environments, used
                       % to drop contents when needed.


% detect inside cluster
\newif\ifinsideproblemcluster

% problem is reset at chapter
\newcounter{problem}[chapter]

% problempart is reset at cluster beginning.
\newcounter{problempart}[problem]


%%%%%%%%%%%%%%%%%%%%%%%%%%%%%%%%%%%%%%%%%%%%%%%%%%%%%%%%%%%%%%%%%%%%%%
%% BEGIN: Diderot Commands
%%%%%%%%%%%%%%%%%%%%%%%%%%%%%%%%%%%%%%%%%%%%%%%%%%%%%%%%%%%%%%%%%%%%%%
\newcommand{\depend}[1]
{}
\newcommand{\solution}
{\noindent\paragraph{Solution.}}

\newcommand{\help}
{\noindent\paragraph{Hint.}}

\newcommand{\explain}
{\noindent\paragraph{Explanation.}}

%%%%%%%%%%%%%%%%%%%%%%%%%%%%%%%%%%%%%%%%%%%%%%%%%%%%%%%%%%%%%%%%%%%%%%
%% END: Diderot Commands
%%%%%%%%%%%%%%%%%%%%%%%%%%%%%%%%%%%%%%%%%%%%%%%%%%%%%%%%%%%%%%%%%%%%%%


%%%%%%%%%%%%%%%%%%%%%%%%%%%%%%%%%%%%%%%%%%%%%%%%%%%%%%%%%%%%%%%%%%%%%%
%% BEGIN: Diderot Environments
%%%%%%%%%%%%%%%%%%%%%%%%%%%%%%%%%%%%%%%%%%%%%%%%%%%%%%%%%%%%%%%%%%%%%%

% wrap inside \[ and \] if targethtml is on
% this seems to work in pandoc even if \targethtml is not defined
% very odd behaviour but it might be because pandoc does not
% handle \ifdefined properly.
\newcommand{\htmlmath}[1]
{
\ifdefined\targethtml
\[
{#1}
\]
\else
{#1}
\fi
}

\NewDocumentEnvironment{diderotfig}{o}
{
\IfNoValueTF{#1}
{\paragraph*{Figure.}}
{\paragraph*{Figure [#1].}}
}
{ 

}

%%%%%%%%%%%%%%%%%%%%%%%%%%%%%%%%%%%%%%%%%%%%%%%%%%%%%%%%%%%%%%%%%%%%%%
%% BEGIN: AMSMath Environments
%%%%%%%%%%%%%%%%%%%%%%%%%%%%%%%%%%%%%%%%%%%%%%%%%%%%%%%%%%%%%%%%%%%%%%


\theoremstyle{definition}
\newtheorem{theorem}{Theorem}[section]
\newtheorem{corollary}[theorem]{Corollary}
\newtheorem{lemma}[theorem]{Lemma}

\newtheorem{definition}{Definition}[section]
\newtheorem{algorithm}[definition]{Algorithm}
\newtheorem{code}[definition]{Pseudo Code}
\newtheorem{costspec}[definition]{Cost Specification}
\newtheorem{datatype}[definition]{Data Type}
\newtheorem{datastr}[definition]{Data Structure}
\newtheorem{example}{Example}[section]
\newtheorem{syntax}[definition]{Syntax}


\theoremstyle{definition}
\newtheorem{exercise}{Exercise}[section]
%\newtheorem{problem}[exercise]{Problem}


\theoremstyle{remark}
\newtheorem*{assumption}{Assumption}
\newtheorem*{hint}{Hint}
\newtheorem*{important}{Important}
\newtheorem*{note}{Note}
\newtheorem*{remark}{Remark}
\newtheorem*{reminder}{Reminder}




%%%%%%%%%%%%%%%%%%%%%%%%%%%%%%%%%%%%%%%%%%%%%%%%%%%%%%%%%%%%%%%%%%%%%%
%% END: AMSMath Environments
%%%%%%%%%%%%%%%%%%%%%%%%%%%%%%%%%%%%%%%%%%%%%%%%%%%%%%%%%%%%%%%%%%%%%%

%%%%%%%%%%%%%%%%%%%%%%%%%%%%%%%%%%%%%%%%%%%%%%%%%%%%%%%%%%%%%%%%%%%%%%
%% BEGIN: Diderot Environments
%%%%%%%%%%%%%%%%%%%%%%%%%%%%%%%%%%%%%%%%%%%%%%%%%%%%%%%%%%%%%%%%%%%%%%
%% \NewDocumentEnvironment{algorithm}{o}
%% {
%% \IfNoValueTF{#1}
%% {\paragraph*{Algorithm.}}
%% {\paragraph*{Algorithm [#1].}}
%% }
%% { 

%% }


%% \NewDocumentEnvironment{costspec}{o}
%% {
%% \IfNoValueTF{#1}
%% {\paragraph*{Cost Specification.}}  
%% {\paragraph*{Cost Specification [#1].}}
%% }
%% { 
%% }

%% \NewDocumentEnvironment{datastr}{o}
%% {
%% \IfNoValueTF{#1}
%% {\paragraph*{Data Structure}} 
%% {\paragraph*{Data Structure [#1].}} 
%% }
%% { 
%% }


%% \NewDocumentEnvironment{datatype}{o}
%% {
%% \IfNoValueTF{#1}
%% {\paragraph*{Datatype.}} 
%% {\paragraph*{Datatype [#1].}} 
%% }
%% { 
%% }

%% \NewDocumentEnvironment{definition}{o}
%% {
%% \IfNoValueTF{#1}
%% {\paragraph*{Definition.}}
%% {\paragraph*{Definition [#1].}}
%% }
%% { 
%% }

%% \NewDocumentEnvironment{example}{o}
%% {
%% \IfNoValueTF{#1}
%% {\paragraph*{Example.}}
%% {\paragraph*{Example [#1].}}
%% }
%% { 
%% }

%% \NewDocumentEnvironment{exercise}{o}
%% {
%% \IfNoValueTF{#1}
%% {\paragraph*{Exercise.}}
%% {\paragraph*{Exercise [#1].}}
%% }
%% { 
%% }

%% BEGIN: Cluster
%% This environment accepts two optional arguments
%% one is a number one is a title
%% number is taken to be points.
%%
%% If there are no arguments, then it produces nothing
%% If there are arguments, then it produces a heading
%% and it separates points and title.
\NewDocumentEnvironment{cluster}{o}
{ 
  \IfNoValueTF{#1}
  {}
  {\paragraph*{#1}}
}
{
}
%% END: cluster

%% BEGIN: mproblem
%% This environment accepts two optional arguments
%% one is a number one is a title
%% number is taken to be points.
%% This is how problem environment should be written as 
%%
%% If there are no arguments, then it produces nothing
%% If there are arguments, then it produces a heading
%% and it separates points and title.
\NewDocumentEnvironment{mproblem}{oo}
{ 
  \insideproblemclustertrue

  \stepcounter{problem}

  \IfNoValueTF{#2}
  {
    \IfNoValueTF{#1}
    {}
    {\IfInteger{#1}
               {\noindent\textbf{Problem \theproblem} (#1 Points)}            
               {\noindent\textbf{Problem \theproblem: #1}}
    }
  }
  {
   \IfInteger{#1}
             {\noindent\textbf{Problem \theproblem: #2} (#1 Points)}
             {\noindent\textbf{Problem \theproblem: #1} (#2 Points)}
  }

}
{
  %% remember that we are inside a cluster 
  \insideproblemclusterfalse


}
%% END: mproblem

%% BEGIN: xchoice list

\newlist{xchoice}{enumerate}{1}
\setlist[xchoice]{label*=\alph*)~,ref=\alph*,leftmargin=2em}

%\makeatletter %% We are in .sty, not needed 
\newcommand\rawchoice{%
    \item\relax
}
\newcommand\rawchoicestar{%
    \item\relax
}
%\makeatother %% We are in .sty, not needed 


\NewDocumentCommand\choice{s}{%
  \IfBooleanTF#1%
    {\rawchoice}% If a star is seen
    {\rawchoicestar}%     If no star is seen
}
%% END: xchoice list

%% TODO Finish
\NewDocumentEnvironment{pickany}{}
{
}
{ 
}


\NewDocumentEnvironment{gram}{o}
{
\IfNoValueTF{#1}
{}
{\paragraph*{#1.}}
}
{ 
}


\NewDocumentEnvironment{flex}{o}
{
\IfNoValueTF{#1}
{\smallskip}
{
\hfill\break
\medskip 
\noindent
{\large \textbf{#1.}}}
}

{ 
\IfNoValueTF{#1}
{\smallskip}
{\medskip}
}

%% 
%% This environment accepts two optional arguments
%% one is a number one is a title
%% number is taken to be points.
%% This is how problem environment should be written as 
\NewDocumentEnvironment{problem}{oo}
{
  %% Step the counter
  %% problem if not within cluster
  %% problempart otherwise

  \ifinsideproblemcluster
    \stepcounter{problempart}   
    \IfNoValueTF{#2}
    {
      \IfNoValueTF{#1}
      {}
      {\IfInteger{#1}
                 {\paragraph{Part \theproblem.\theproblempart: (#1 Points)}}            
                 {\paragraph{Part \theproblem.\theproblempart: #1}}
      }
    }
    {
     \IfInteger{#1}
               {\paragraph{Part \theproblem.\theproblempart: #2 (#1 Points)}}
               {\paragraph{Part \theproblem.\theproblempart: #1 (#2 Points)}}
    }
  \else
    \stepcounter{problem}  
    \IfNoValueTF{#2}
    {
      \IfNoValueTF{#1}
      {}
      {\IfInteger{#1}
                 {\paragraph{Problem \theproblem: (#1 Points)}}            
                 {\paragraph{Problem \theproblem: #1}}
      }
    }
    {
     \IfInteger{#1}
               {\paragraph{Problem \theproblem: #2 (#1 Points)}}
               {\paragraph{Problem \theproblem: #1 (#2 Points)}}
    }
  \fi
}
{     
}
%% \NewDocumentEnvironment{important}{o}
%% {
%% \IfNoValueTF{#1}
%% {\paragraph*{Important.}}
%% {\paragraph*{Important [#1].}}
%% }
%% { 

%% }

%% \NewDocumentEnvironment{lemma}{o}
%% {
%% \IfNoValueTF{#1}
%% {\paragraph*{Lemma.}}
%% {\paragraph*{Lemma [#1].}}
%% }
%% { 

%% }


%% \NewDocumentEnvironment{note}{o}
%% {
%% \IfNoValueTF{#1}
%% {\paragraph*{Note.}}  
%% {\paragraph*{Note [#1].}}  
%% }
%% { 
%% }

\NewDocumentEnvironment{preamble}{o}
{
\IfNoValueTF{#1}
%{\paragraph*{Preamble.}}  
{}
{\paragraph*{#1.}}  
}
{ 
}

\NewDocumentEnvironment{proposition}{o}
{
\IfNoValueTF{#1}
{\paragraph*{Proposition.}}
{\paragraph*{Proposition [#1].}}

}
{ 

}


%% \NewDocumentEnvironment{remark}{o}
%% {
%% \IfNoValueTF{#1}
%% {\paragraph*{Remark.}}
%% {\paragraph*{Remark [#1].}}
%% }
%% { 
%% }

%% \NewDocumentEnvironment{solution}{o}
%% {
%% \IfNoValueTF{#1}
%% {\paragraph*{Solution.}}
%% {\paragraph*{Solution [#1].}}
%% }
%% { 

%% }

%% \NewDocumentEnvironment{syntax}{o}
%% {
%% \IfNoValueTF{#1}
%% {\paragraph*{Syntax.}} 
%% {\paragraph*{Syntax [#1].}} 
%% }
%% { 

%% }

%% \NewDocumentEnvironment{theorem}{o}
%% {
%% \IfNoValueTF{#1}
%% {\paragraph*{Theorem.}}
%% {\paragraph*{Theorem [#1].}}
%% }
%% { 

%% }

\ifdefined\teachingnotes
\NewDocumentEnvironment{teachnote}{o}
{
%Drop \BODY
}
{
}
\else
\NewEnviron{teachnote}{}  % This drops body.
\fi
\ifdefined\teachingnotes
\NewDocumentEnvironment{teachask}{o}
{
%Drop \BODY
}
{
}
\else
\NewEnviron{teachask}{}  % This drops body.
\fi

%%%%%%%%%%%%%%%%%%%%%%%%%%%%%%%%%%%%%%%%%%%%%%%%%%%%%%%%%%%%%%%%%%%%%%
%% END: Diderot Environments
%%%%%%%%%%%%%%%%%%%%%%%%%%%%%%%%%%%%%%%%%%%%%%%%%%%%%%%%%%%%%%%%%%%%%%2
