\chapter*{Syllabus}
\label{ch:syllabus}

\section{Course Information}

Welcome to 15-897, a graduate course on parallelism and concurrency. 
%
The lectures will takes place Mondays and Wednesdays 10:30-11:50am and it will be at Gates 4303. 

Parallelism and concurrency play a  fundamental role in computing.
%
Due to increasing demand in computational resources, many interesting tasks today, ranging from web servers, to scientific simulations exploit parallelism and concurrency.
%
This course aims to develop a mathematical foundation upon which parallelism and concurrency may be studied and covers a number of fundamental algorithmic and programming techniques.  
%
 


\section{Prerequisites}
There are no specific prerequisites for the course but graduate-level mathematical reasoning techniques, algorithms, and basic programming skills are assumed.
%

\section{Learning Objectives}

The primary objective of the course is to establish a foundation for reasoning about parallelism and concurrency and from this perspective cover a variety of parallel algorithms and techniques.
%
Secondary objectives include
\begin{itemize}
\item Learn about formal models of parallel computation.
\item Learn about the fundamental techniques for designing and analyzing parallel algorithms.
\item Learn about concurrency mechanism and their use in computation. 
\end{itemize}


\section{Diderot}
We will use Diderot as the main hub for course activities.
%
All of the course content, including the schedule, lecture notes, etc will 
will be published on Diderot. 
%
Diderot will also serve as the main communication hub and
announcements and discussions will be conducted through Diderot.
%

\begin{gram}[Posting on Diderot]
Diderot provides several different kinds of posts, including \defn{Question}, \defn{Feedback}, and \defn{Social} posts.
%
For example, you can use \defn{Feedback} posts to send feedback about course content and \defn{Question} content to ask questions.
%
Feedback posts are designed to be private (and anonymous) only but questions can be both public and private.
%

Because this is a graduate course, our goal will be to engage in discussions, rather than question-answer dialog.
%
We will therefore only use \emph{public} Question posts for discussions.
%
You are welcome to use private Question posts for personal matters, and Feedback posts if you would like to send feedback anonymously. 
\end{gram}

\section{Resources}

We will cover a variety of material from a variety of sources, much of which is based on research.
%
There is no standard text for this course.
%
But we will create a set of notes via scribing, which we hope will create a blueprint for further development.


\section{Scribing Notes}

We will have a  designated scribe for each lecture.
%
The scribe will be responsible for taking notes during lecture and working with me afterwards to make an initial draft ready within one week, and a more polished version in two weeks time.
%
To help guide the process, I will try to create an initial outline for each lecture, which I hope will guide the process.

\section{Grading}

Your grade will depend on 
\begin{itemize}
\item attendance and participation in class and on Diderot,
\item fulfillment of scribing responsibilities,
\item assignments ,
\item in class quizzes,
\item class project.
\end{itemize}


\section{Collaboration, Credits, and Citations}


Collaboration is at the heart of learning.  We therefore encourage you to collaborate with your classmates.
%
But there are some limits to collaboration: it should never involve copying actual written material of any sort from another student.
%
This means that your collaborations must remain at the level of discussions, drawings, and etc, and you individually must produce your own work.
%
If for any piece of work, you have collaborated with others, we ask you to credit them in the a specially designated ``Acknowledgments'' section of your work and describe in specific terms the nature of the collaboration.


In addition to personal collaborations, you will also find it useful to read other texts, papers, etc.
%
If you have used such an external resource, then please cite the resource and credit  the authors.
%
This is important not purely from an academic integrity point of view, but also from a strength point of view:
%
building your work on the work of others only increases their quality, and acknowledging them in the process, makes it possible for others to see and understand the connections, and thus to appreciate your work.



\section{Academic Integrity}
We take academic integrity very seriously and will do our best to make the course fair for everyone. 
%
Please read Carnegie Mellon University 
%
\href{https://www.cmu.edu/policies/student-and-student-life/academic-integrity.html}
{Policy on Academic Integrity}. 


%% \begin{example}[Cheating]

%% As instructors, we see many examples of cheating, some of which are colorful and even entertaining.  
%% %
%% Beware that consequences of cheating can be quite severe.
%% %

%% \begin{itemize}

%% \item Copying (partially or wholly) material from another student, e.g., an exam, or in an assignment.

%% \item Sharing your written work, at any stage of completion, with another student.

%% \item Not crediting help received.

%% \item Not citing related works that you have borrowed from.
%% \end{itemize}
%% \end{example}

\section{Accommodations and Make-Up Policy}

\begin{gram}
Please contact the instructor if you have an  approved accommodations from the Disability Resources Center. 

No make-up quizzes, exams, or assignments will be administered, except in the case of documented medical or family emergencies, or other university approved absences. The common cold and other illnesses that does not require hospitalization, unfortunately, do not qualify as an excused absence.

Please make a note of the important dates and do not schedule events such as interviews during those times. We will not administer a make-up exam because it clashes with an interview. Companies should respect your exam schedules and give you various options for interview dates.
\end{gram}
 
\section{Well-Being and Happiness}
\begin{gram}
We care about you and  your happiness and are  always available to talk.
%
Also be aware that 
%
\href{http://www.cmu.edu/counseling/}{Conseling and Psychological Services (CaPS)}
%
offers many services including therapy and crisis management. 
%
When experiencing difficulties, please keep in mind that your problems
may appear smaller to you than they actually are, and reaching out to talk with somebody can make a huge difference.
%
\end{gram}

\section{Safety}

Do not hesitate to call 
%
\href{https://www.cmu.edu/police/}{CMU police}
%
in emergency situations or for other services that they provide.
%
CMU police can also be reached at 412-268-2323.
%
 
